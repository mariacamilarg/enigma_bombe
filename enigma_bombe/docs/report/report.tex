\documentclass[12pt]{article}

\usepackage{graphicx}
\usepackage{epstopdf}
\usepackage[english]{babel}
\usepackage[latin5]{inputenc}
\PassOptionsToPackage{hyphens}{url}\usepackage{hyperref}
\usepackage[left=3cm,top=3cm,right=3cm,bottom=3cm,nohead,nofoot]{geometry}
\usepackage{datenumber}
\usepackage{listings}

\begin{document}

\begin{center}
\Huge
Project: Enigma Machine and Turing-Welchman Bombe

\vspace{10mm}
\Large
Maria Camila REMOLINA GUTI�RREZ
\large
maria.remolina\_gutierrez@telecom-sudparis.eu

\vspace{5mm}
\Large
Advisor: Prof. Eric RENAULT

\vspace{5mm}
\normalsize
\today
\end{center}

\vspace{5mm}
\begin{abstract}
	This is the work proposal to follow in the course Project as part of the master M1 in Computer Science and Communication Networks at T�l�com SudParis. The goal is to implement the Enigma machine used in World War II, followed by the Bombe machine that breaks the cipher, created by Alan Turing and Gordon Welchman at Bletchley Park.
\end{abstract}


\section{Introduction}

The enigma machine is a cipher machine used by the Nazi Germany during World War II in order to send secret coded messages. It was initially a commercial machine bought by banks are businesses. But then the military took it and added an extra security layer called plugboard. It was innovative at the time because it was not a substitution cipher, i.e. the same letter can get different results after encryption. \\

The machine works by a combination of moving rotors and inside wiring as seen in Fig. \ref{fig:enigma_machine}. The way to use it is that when the sender types a message, a bulb lights up indicating the correspondent coded letter. Then, in order to decode, the receiver types the coded message he received and the initial message lights up on the board. In war times the coded messages were transmitted over morse code. \\

\begin{figure}[h!]
	\centering
	\includegraphics[width=0.5\textwidth]{../img/enigma_machine}
	\caption{Enigma Machine - Military Edition}
	\label{fig:enigma_machine}
\end{figure}

In the military version of the Enigma Machine there were 5 possible rotors to pick from, each with 26 possible positions, then there were 10 possible switching in the plugboard that could choose a pair from the available 26 letters. That accounts for $158962555217826360000$ different combinations for
the setting of the army Enigma Machine \cite{numberphile1}. The initial setting for each day was given to the bases in a sheet of paper that had the monthly configurations of the machine. So they changed the setting everyday. \\

So in order to decipher the code, you needed either to have the code sheet or to break the message. This latter was what the scientists at Bletchley Park did; and what I will try to recreate in this project. 

\section{General Goal}

To understand the code breaking process behind the enigma cipher.


\section{Specific Goals}

\begin{itemize}
	\item To implement the Enigma machine with software
	\item To implement the Bombe machine that breaks the Enigma cipher
	\item To understand the mathematical and probabilistic techniques used to break a cipher with limited time and computational resources. 
	\item To understand the weaknesses exploited that allowed to break the Enigma cipher
\end{itemize}


\section{Implementation}

I implemented a project in the Python programming language that replicates both the Enigma Machine and the Bombe Machine. I explain in the following subsections the details of each of them.

\subsection{Enigma Machine}

There are different models of the Enigma machine that were developed through time. In this project I chose one of the latest, i.e. with more complicated settings. This is the model M3 \& M4 Naval (from Frebuary 1942) that has 8 possible rotors \cite{dade,wikirotors}. 

\subsubsection{Enigma's algorithm}
The enigma machine consists on a set of physical connections that allow electrical signals to travel through inner cross wirings that move place every time a key is pressed. In the most simple form we could express that any typed key in the keyboard is transformed to another letter that lights up in the bulbs panel. So we could say: 

$$\rm{SIGNAL_{OUT}} = P( S ( P(\rm{SIGNAL_{IN}}) ) )$$

$$S(\rm{SIGNAL}) = R_1(R_2(R_3(F(R_3(R_2(R_1(\rm{SIGNAL})))))))$$

Where $P$ represents the Plugboard; $S$ represents the Set of rotors an reflectors; $R_i$ represents the i-th rotor (in this case there are 3 but there could be more); and $F$ represents the reflector. All of this process can be graphically understood with the image below (Fig. \ref{fig:wiring}): 

\begin{figure}[h!]
	\centering
	\includegraphics[width=0.5\textwidth]{../img/wiring_example}
	\caption{Enigma Machine Wiring}
	\label{fig:wiring}
\end{figure}

The machine has a symmetric property so in order to decipher the message one has to put the same rotor settings as when the person typing the message started. So this leads us to the configuration of the machine, i.e. the input. 

\subsubsection{Input}
The machine has different settings that changed each day and that were communicated via code sheets like the one in Fig. \ref{fig:code_sheet}. 

\begin{figure}[h!]
	\centering
	\includegraphics[width=\textwidth]{../img/code_sheet_example}
	\caption{Enigma Machine Wiring}
	\label{fig:code_sheet}
\end{figure}

I handle this as a configuration text file in my implementation. I will describe the components as follows: 

\begin{itemize}
	\item Rotor selection and order: Depending on the machine there are up to 8 different rotors (I-VIII) from which to choose 3 or 4. So an input is which of them and in which permutation from left to right. Example - V IV I.
	\item Rotor's ring settings (\textit{Ringstellung}): Within each rotor there is an internal ring that maps position to a letter, but it can be switched so that the internal wiring mapping is offset. Example - N D V.
	\item Rotor's initial positions (\textit{Grundstellung}): Within the machine, when locating the rotors, they have to be in a particular initial position. This is achieved by inserting the rotor and turning it to the respective starting point. Example: D Y A.
	\item Plugboard pairs: This corresponds to the symmetric correspondence of letters in the other plugboard panel. They are typically 10 pairs. 
	\item Reflector selection: As with the rotors, there are 4 different reflectors from which to choose, all with different cross-connections. 
\end{itemize}

Apart from all of these values that changes from user to user, I also have an input configuration file that loads the historically accurate wiring of each reflector and rotor. For the rotors there is an additional parameter called the turnover point. The rotors work as a clock in the way they rotate. The right-most represent the second, the middle the minutes, and the left-most the hours. However they don't turn all in the same letter ('Z' as one would expect), they all each have their own turnover point that triggers its right partner rotation. Those turnovers are obtained from \cite{wiki}. Finally the message to cipher is also input as a text file. \\

\subsubsection{Processing}
As for the processing of the plain text, here is a pseudo code with the simplified process:

\begin{lstlisting}[language=Python]

for letter in text:

	# advance rotors according to their own position,
	# and turnover points
	step_rotors()
	
	# enter plugboard switch, if no correspondance there 
	# is no connection, so letter doesn't change
	plugboard_letter = plugboard.get(letter, letter)
	
	# enter rotors (inwards - right to left)
	rotor_names_inverted = rotor_names[::-1]
	pin_in = ALPHABET.index(plugboard_letter)
	
	for rotor_name in rotor_names_inverted:
		rotor = rotors[rotor_name]
		pin_in = rotor.process_inwards(pin_in)
	
	# enter reflector
	pin_out = reflector.reflect(pin_in)
	
	# enter rotors (outwards - left to right)
	for rotor_name in rotor_names:
		rotor = rotors[rotor_name]
		pin_out = rotor.process_outwards(pin_out)
	
	# enter plugboard switch
	rotor_letter = ALPHABET[pin_out]
	
	# calculate ciphered letter
	ciphered_letter = plugboard.get(rotor_letter, 
					rotor_letter)
	
\end{lstlisting}

Somethings to be aware of in the processing are that the rotors step before ciphering, not after. Also, the turnover points need to be checked each time the rotors step, some of them  even have 2 different turnovers. 

\subsubsection{Output}
The output of the machine is straightforward, it correspond to the cipher letter of each key press after going through the transformations. And for a whole text is just the concatenation of those letters. If this output is to be typed again in the machine with the same initial configuration, the operator on the other side of the line will obtain the original plain message. \\

\subsubsection{Challenges}

The biggest challenge in implementing the Enigma machine is knowing exactly how it worked. Even thought there is a great amount of information online, the depth of it is not that significant. Most of the resources tend to shallowly explain the behavior but there are several subtleties and details that were hard to find. I put in the references the most trustworthy and complete information I found on the matter \cite{rijmenants,sale,cryptomuseum,wiki,wikicryptanalysis,ellsbury}. \\

Another challenge was the translation of this information, because the original machine settings are in German. This implies that there are multiple acceptable translations for several parts of the machine that might be opposite among different sources. A concrete example of this goes within the rotor, where there are 2 different settings called \textit{Ringstellung} and \textit{Grundstellung}. They refer to the ring setting within each the rotor and to the ground setting or offset of the rotor within the machine, respectively. However figuring out which one was which presented a problem because different sources referred to them by different names, sometimes even opposite. So I had to solve this by recurring to the German words. \\

Finally, I would like to refer as well to different simulators that are found on the internet in which I could get a phenomenological understanding of the machine. I was able to try out different configurations and learning its way to function. I also confirmed my solution against these simulators in order to test my implementation. This was a positive match. 

\begin{itemize}
	\item Cryptii Simulator: \url{https://cryptii.com/pipes/enigma-machine}
	\item Louise Dade Simulator: \url{http://enigma.louisedade.co.uk/enigma.html}
\end{itemize}



\subsection{Bombe Machine}

\subsubsection{Enigma Exploits}

In order to break the enigma, mathematicians Alan Turing and Gordon Welchman, with the collaboration of the staff at Bletchley Park, design the Bombe Machine. It exploited the Enigma Machine's flaws to crack the Nazi's communications. However, the machine itself was not all, it exploited behavioral patterns in the communication as well. Here I list the main weaknesses that lead to deciphering the Enigma:

\begin{itemize}
	\item The biggest flaw of the Enigma Machine is that because of the internal wiring and design, a letter can never be encrypted to itself. This opened the door to cryptoanalysts to locate possible message deciphers. \cite{numberphile2}
	\item Another breakthrough was when they realized about specific patterns that would repeat themselves in a conversation. For example the daily weather forecast, the heil Hitler salute, or war jargon. These parts of the conversation tipically followed a standard format that allows to create "cribs", an intelligent guess of part of the code. \cite{computerphile}
	% ex: "weather for the night" "situation eastern channel" "beacons lit as ordered" etc
	\item In addition to the previous facts, sometimes reckless operators forgot to change the daily settings, which lower the odds of combinations and gave more data to the analysts.
	
\end{itemize}

\subsubsection{Bombe's algorithm}

So the idea of the Bombe is to try have several enigma machines in parallel (as seen in Fig. \ref{fig:bombe_machine}), each one testing its own setting, but in a smart and optimizing manner. Utilizing all the previous known exploits, as well as mathematical inferences and principles of conditional sequential bayesian probability. \\

\begin{figure}[h!]
	\centering
	\includegraphics[width=0.7\textwidth]{../img/bombe_machine}
	\caption{Bombe Machine at Bletchley Park}
	\label{fig:bombe_machine}
\end{figure}

To begin we need a \textit{crib}, that is a message and its most-likely cipher. This was obtained by comparing the common phrases and finding where in the cipher text no letter maps to itself. With the crib we create then a hypothesis called a \textit{menu}, an analog of a computer program for the Bombe. The menu represents a graph where each links connects 2 nodes that can be translated to each other via the enigma, in that particular position. Fig \ref{fig:map} displays what the crib and map would be for our example in the project. 

\begin{figure}[h!]
	\centering
	\includegraphics[width=0.75\textwidth]{../img/map}
	\caption{Bombe Map Graph}
	\label{fig:map}
\end{figure}

Now we need to find an enigma configuration that corresponds to the menu, and here is where the plugboard comes into place. The Bombe starts by taking a letter of the graph, called input and making a guess of its corresponding pair in the plugboard (ex: $P(T)=A$). From there, it starts to follow the paths on the graph, specially the loops, and starts finding that the guess implies other connections in the board. If we end up with a contradiction then our guess is wrong and we move to the next guess. If by the end all the guesses are wrong, then the rotor configuration is wrong. \\

Alan Turing was able to decrease the time by proving that if the Bombe finds a contradiction, then all the previous guesses are also wrong, so the machine doesn't have to check them again. He also design the machine so that the circuits could calculate the correspondences immediately. \\

Let's look at an example of one of these possible deductions:

\begin{enumerate}
	\item I suppose $P(T)=A$
	\item I know that $M = P(S_0(P(T)))$ because of the crib
	\item As P is symmetrical, I can apply it to both sides so $P(M) = S_0(P(T))$ 
	\item Using my guess I find that $P(M) = S_0(A)$ 
	\item According to my enigma replica, that is $P(M) = F$, so now I have another plugboard setting. 
	\item Following the loop and doing the same operations, I find that $I = P(S_6(P(M)))$
	\item Applying P on both sides $P(I) = S_6(P(M))$
	\item Using the previously found setting $P(I) = S_6(F)$
	\item Using the enigma machine $P(I) = Y$, and that is my third plugboard setting
	\item Now to close the loop I also know that $T = P(S_{14}(P(I)))$
	\item Applying P on both sides $P(T) = S_{14}(P(I))$
	\item Using the plugboard $P(T) = S_{14}(Y)$
	\item Using the enigma machine $P(T) = K$, BUT here lies a problem, I guessed $P(T)=A$ so I found a contradiction
\end{enumerate}

Here all the previous guesses are now invalid for the current rotor configuration and I have to go to the next guess and repeat. As seen in the example, and proved my Alan Turing, the most loops the easier it is to find contradictions, i.e. the less possible plugboards to check. Here is a table (Fig. \ref{fig:loops})that shows this relation \cite{wikibombe}.\\

\begin{figure}[h!]
	\centering
	\includegraphics[width=\textwidth]{../img/loops}
	\caption{Estimated number of bombe stops per rotor order}
	\label{fig:loops}
\end{figure}

Now, when a plugboard is consistent it becomes a candidate and it was then checked by hand at the park. 

\subsubsection{Input}
I handle the input as a configuration text file in my implementation. I will describe the components as follows: 

\begin{itemize}
	\item Crib: The piece of text plain and ciphered that was selected as a candidate by the staff.
	\item Menu paths: The menu paths in the order that the program will follow, and create the deductions. It starts with the input letter and it it preferable to have all the loops at first, in order to reject combinations faster. 
	\item Rotor's selection and order: This because the Bombe runs several rotor combinations in parallel, so in my implementation I input the rotors. They had several Bombes working at the same time as well. 
	\item Initial rotor ring settings and reflector: This is for simplicity of the program. Again, in the big Bombe all the combinations are tested. 
\end{itemize}


\subsubsection{Processing}
The processing implements all the algorithm explained before within the function \texttt{run()}. It is important to remark the high use of Python dictionaries (i.e. hash tables) as a data structure. They make the access to the plugboards and settings straightforward, as opposed to searching on a list. 

\subsubsection{Output}
The output of the machine is all the possible plugboards associated to each rotor configuration. This is output to a file that was then checked by the staff. They translated the message using those configurations and see which one made sense. From the example we can see that the program outputs DZR and the plugboard that the machine used to cipher. So it cracked it. \\

\subsubsection{Challenges}

As well as for the Enigma, the biggest challenge in implementing the Bombe machine is knowing exactly how it worked. As opposed to Enigma that was a commercial machine and that had over $100000$ replicas, the Bombe information was kept secret for a very long time. There are no more than 200 Bombes in the world and less than 5 are rebuilt and operational. Most of the information for the plans was kept classified until 2010, not even 10 years ago. So it was very challenging to get to the nuts and bolts of the apparatus. However, when combining all the reliable sources I was able to make this compendium and to understand how it worked. I put in the references the most trustworthy and complete information \cite{ellsbury,wikibombe,leblob,101computing,salebombe,carter,oberzalek,wikibanburismus}. \\

Another remark that also opposes to the Enigma is the lenght of the Bombe. There are much more steps deductions and in general code that needs to be written. While implementing it, it is evident why the Bombe was so much more complicated, expensive and large. \\

Finally, and as in the previous machine, I got some insights from two Bombe simulators online. However as the is very complex, using them is not as straightforward and required a lot of learning. After all, each bomb had several operators. So I just had a glimpse on their configuration files, specially the menu input. Here are the references: 

\begin{itemize}
	\item Magnus Ekhall \& Fredrik Hallenberg Simulator: \url{http://www.lysator.liu.se/~koma/turingbombe/} \cite{ekhall}
	\item 101 Computing Simulator: \url{https://www.101computing.net/turing-welchman-bombe/}
\end{itemize}


\subsection{Code}
All of the algorithms, inputs, outputs and processing were coded in Python. The code of both machines is available in the project attached to this report. The structure can be seen in Fig. \ref{fig:code_structure}.\\

\begin{figure}[h!]
	\centering
	\includegraphics[width=0.4\textwidth]{../img/code_structure}
	\caption{Code Structure}
	\label{fig:code_structure}
\end{figure}

There is a program called \texttt{example.py} that shows how to use the programs and when executed, outputs the result of both enigma and bombe machines to their respective results files. It also outputs to console like Fig. \ref{fig:code_stdout}. 

\begin{figure}[h!]
	\centering
	\includegraphics[width=0.8\textwidth]{../img/code_stdout}
	\caption{Example Code Standard Output}
	\label{fig:code_stdout}
\end{figure}


\section{Conclusions}

From this  project I was able to make the following conclusions and achievements: 

\begin{itemize}
	\item I was able to understand to a deep level and to replicate the code ciphering and code breaking process behind the Enigma and Bombe machines. I am personally amazed by the ability of cryptoanalysts to find a door to break what everyone thought was impossible.
	\item I was able to  understand the effect of mathematics in this particular case. I understood the probabilites and the deduction techniques in order to find a viable solution. This is in my opinion one of the most relevant examples in history were mathematicians can save lives. It's just another proof that research in basic sciences are the foundation for application way beyond the current scope of humanity. 
	\item I was able to understand than even though deciphering the message was a very hard task, in order to win the war you needed much more. There were a lot of difficult decisions that had to be made so that the Nazis didn't realize the machine was cracked. A lot of lives that were sacrificed on purpose in order to win. The strategy goes beyond the cipher.  
	\item I conclude that is by implementing that you deeply understand something. I had and idea about the machines, but it is when trying to replicate them that you start to see all the subtleties, details and complexities that rely within. This was a very fructiferous project to make.
\end{itemize}

\subsection{Future Work}
The future work I would like to perform after this class is to improve the Bombe so that it runs in parallel (just as the machines at Bletchley Park). I would like to learn deeper about the physical wirings and connections inside the Bombe that made it so fast. Then, it would be really interesting to create a physical system with current electronics and compare the times. 


\begin{thebibliography}{10}
% -- enigma --
	\bibitem{hodges} A. Hodges. \textit{Alan Turing: The Enigma}. Princeton, N.J: Princeton University Press (2012). 
	\bibitem{numberphile1} Numberphile. \textit{158,962,555,217,826,360,000 (Enigma Machine)}. YouTube (2013). % https://www.youtube.com/watch?v=G2_Q9FoD-oQ
	\bibitem{rijmenants} D. Rijmenants. \textit{Technical Details of the Enigma Machine}. Obtained from: \url{http://users.telenet.be/d.rijmenants/en/enigmatech.htm}
	\bibitem{sale} T. Sale. \textit{Military Use of the Enigma}. Obtained from: \url{https://www.codesandciphers.org.uk/enigma/enigma3.htm}
	\bibitem{cryptomuseum} Cryptomuseum. \textit{Enigma Cipher Machines}. Obtained from: \url{https://www.cryptomuseum.com/crypto/enigma/index.htm}	
	\bibitem{wiki} Wikipedia. \textit{Enigma Machine}. Obtained from: \url{https://en.wikipedia.org/wiki/Enigma_machine}
	\bibitem{wikicryptanalysis} Wikipedia. \textit{Cryptanalysis of the Enigma}. Obtained from: \url{https://en.wikipedia.org/wiki/Cryptanalysis_of_the_Enigma}
	
	\bibitem{dade} L. Dade. \textit{Enigma Machine: How It Works}. Obtained from: \url{http://enigma.louisedade.co.uk/howitworks.html}
	\bibitem{wikirotors} Wikipedia. \textit{Enigma Rotor Details}. Obtained from: \url{https://en.wikipedia.org/wiki/Enigma_rotor_details}
	
% -- bombe --
	\bibitem{ellsbury} G. Ellsbury. \textit{The Enigma and The Bombe}. Obtained from: \url{http://www.ellsbury.com/enigmabombe.htm}
	\bibitem{numberphile2} Numberphile. \textit{Flaw in the Enigma Code}. YouTube (2013). % https://www.youtube.com/watch?v=V4V2bpZlqx8
	\bibitem{computerphile} Computerphile. \textit{Tackling Enigma (Turing's Enigma Problem)}. YouTube (2013). % https://www.youtube.com/watch?v=kj_7Jc1mS9k
	\bibitem{ekhall} M. Ekhall, F. Hallenberg. \textit{US Navy Cryptanalytic Bombe - A Theory of Operation and Computer Simulation}. Proceedings of the 1st Conference on Historical Cryptology, pages 103-108,	Uppsala, Sweden, 18-20 June, 2018.
	\bibitem{wikibombe} Wikipedia. \textit{Bombe}. Obtained from: \url{https://en.wikipedia.org/wiki/Bombe}
	\bibitem{leblob} Le Blob. \textit{La "bombe" de Turing : vers le d�cryptage industriel}. YouTube (2017). % https://www.youtube.com/watch?v=2dKG21u2aSo
	\bibitem{101computing} 101 Computing. \textit{Enigma Crib Analysis}. Obtained from: \url{https://www.101computing.net/enigma-crib-analysis/}
	\bibitem{salebombe} T. Sale. \textit{Virtual Wartime Bletchley Park}. Obtained from: \url{http://www.codesandciphers.org.uk/virtualbp/tbombe/tbombe.htm}
	\bibitem{carter} F. Carter. \textit{The Turing Bombe}. Rutherford Journal. Obtained from: \url{http://www.rutherfordjournal.org/article030108.html}
	\bibitem{oberzalek} M. Oberzalek. \textit{Breaking the Enigma}. Obtained from: \url{http://www.mlb.co.jp/linux/science/genigma/enigma-referat/node6.html}	
	\bibitem{wikibanburismus} Wikipedia. \textit{Banburismus}. Obtained from: \url{https://en.wikipedia.org/wiki/Banburismus}
	
	
\end{thebibliography}


\end{document} 